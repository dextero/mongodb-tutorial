\documentclass{beamer}

\usepackage{amsmath}
\usepackage{amsfonts}
\usepackage{amssymb}
\usepackage{graphicx}
\usepackage{listings}
\usepackage{color}

\hypersetup{
  colorlinks = true,
  linkcolor = blue,
  urlcolor = blue,
  citecolor = blue,
  anchorcolor = blue
}

\definecolor{lightgray}{rgb}{.9,.9,.9}
\definecolor{darkgray}{rgb}{.4,.4,.4}
\definecolor{darkgreen}{rgb}{0,0.4,0}

\lstdefinelanguage{JavaScript}{
    keywords={typeof, new, true, false, catch, function, return, null, catch, switch, var, if, in, while, do, else, case, break, \$lt, \$gt, \$lte, \$gte, \$ne, \$in, \$nin, \$and, \$or, \$nor, \$not, \$elemMatch, \$set, \$unset, \$rename, \$inc, \$mul, \$min, \$max, \$push, \$pop, \$pull},
    keywordstyle=\color{blue}\bfseries,
    ndkeywords={class, export, boolean, throw, implements, import, this},
    ndkeywordstyle=\color{darkgray}\bfseries,
    identifierstyle=\color{black},
    sensitive=false,
    comment=[l]{//},
    morecomment=[s]{/*}{*/},
    commentstyle=\color{darkgreen}\ttfamily,
    stringstyle=\color{red}\ttfamily,
    morestring=[b]',
    morestring=[b]"
}

\lstset{
    language=JavaScript,
    backgroundcolor=\color{lightgray},
    extendedchars=true,
    basicstyle=\tiny\ttfamily,
    showstringspaces=false,
    showspaces=false,
    tabsize=2,
    breaklines=true,
    showtabs=false,
    captionpos=b
}

\usepackage{polski}
\usepackage[polish]{babel}
\usepackage[utf8]{inputenc}

\begin{document}

  \begin{frame}
    \begin{center}
      \includegraphics[width=\textwidth,keepaspectratio]{img/logo.png}

      Dawid Czech, Marcin Radomski
    \end{center}
  \end{frame}

  \begin{frame}
    \frametitle{Model danych}
    \begin{columns}
      \column{.6\textwidth}
      \includegraphics[scale=0.5]{img/db-structure.png}

      \column{.4\textwidth}
      Dane w MongoDB są przechowywane jako zbiór \emph{par klucz-wartość}, pogrupowanych w \emph{dokumenty}. Wiele dokumentów tworzy \emph{kolekcję}, a zbiór kolekcji to właściwa \emph{baza}.
    \end{columns}
  \end{frame}

  \begin{frame}
    \frametitle{Dokument}
    Dane znajdujące się w dokumencie są zapisywane jako obiekty w formacie BSON (Binary JSON), który definiuje nie tylko typy prymitywne jak liczba czy tekst, ale również tablice czy podobiekty:
    \lstinputlisting{listings/document.js}

    Więcej o formacie BSON: \url{http://bsonspec.org/}
  \end{frame}

  \begin{frame}
    \frametitle{Zalety}
    \begin{itemize}
      \item Łatwość tworzenia bazy
      \begin{itemize}
        \item Odwołanie do nieistniejącej kolekcji tworzy ją automatycznie
      \end{itemize}
      \item Brak określonej struktury danych w dokumencie
      \item Intuicyjna struktura danych w bazie
      \begin{itemize}
        \item Tablice, słowniki, nawet wyrażenia regularne
      \end{itemize}
      \item Konsola z interpreterem JavaScriptu
    \end{itemize}
  \end{frame}

  \begin{frame}
    \frametitle{Wady}
    \begin{itemize}
      \item Nie nadaje się do każdego rodzaju danych
      \item Brak określonej struktury danych w dokumencie...
      \begin{itemize}
        \item Konieczność używania indeksów
		\item Łatwość popełnienia błędu
      \end{itemize}
      \item Konieczność przechowywania nazw pól w każdym dokumencie
      \begin{itemize}
        \item Ogromne marnotrawstwo pamięci w przypadku dużych baz
      \end{itemize}
      \item Duplikacja danych
      \item Niecodzienna składnia porównań w zapytaniach
    \end{itemize}
  \end{frame}

  \begin{frame}
    \frametitle{Dobry przykład}
  \end{frame}

  \begin{frame}
    \frametitle{Zły przykład}
  \end{frame}

  \begin{frame}
    \frametitle{CRUD}
    \begin{itemize}
      \item Select: \texttt{db.kolekcja.find(FILTR)}
      \lstinputlisting{listings/crud-select.js}
      \item Insert: \texttt{db.kolekcja.insert(DOKUMENT)}
      \lstinputlisting{listings/crud-insert.js}
      \item Update: \texttt{db.kolekcja.update(FILTR, OPERACJA)}
      \lstinputlisting{listings/crud-update.js}
      \item Remove: \texttt{db.kolekcja.remove(FILTR)}
      \lstinputlisting{listings/crud-remove.js}
    \end{itemize}
  \end{frame}

  \begin{frame}
    \frametitle{Filtrowanie dokumentów}
    \begin{itemize}
      \item \texttt{\{ klucz: 'wartość' \}} - 'jest równe'
      \item \texttt{\{ \$operator: argument(y) \}}
      \begin{itemize}
        \item \texttt{\{ \$exists: 1 \}} - można podać 0, żeby wybrać elementy nie zawierające klucza
        \item \texttt{\{ \$lt: 20 \}} - 'mniejsze od'
        \item \texttt{\$gt, \$lte, \$gte, \$ne} - inne porównania
        \item \texttt{\{ \$in: [ 'A', 'B' ] \}} - 'należy do zbioru'
        \item \texttt{\$nin} - 'nie należy do zbioru'
      \end{itemize}
      \item Operatory logiczne: \texttt{\$and, \$or, \$nor, \$not}
      \item Do stringów - wyrażenia regularne: \texttt{\{ name: /.*a.*/ \}}
    \end{itemize}
    \lstinputlisting{listings/find-basic.js}
  \end{frame}

  \begin{frame}
    \frametitle{Filtrowanie dokumentów - tablice i subdokumenty}
    \begin{itemize}
      \item Tablice
      \begin{itemize}
        \item \texttt{\{ tablica: [ 'A', 'B' ] \}} - dokładne dopasowanie
        \item \texttt{\{ tablica: 'A' \}} - czy tablica zawiera element 'A'?
        \item \texttt{\{ 'tablica.2': 'A' \}} - czy trzeci element tablicy to 'A'? (indeksowanie od 0)
        \item \texttt{\{ \$elemMatch: warunek \}} - wybieranie niektórych elementów tablicy
      \end{itemize}
      \item \texttt{\{ 'kot.kolor': 'czarny' \}} - sprawdzanie pola subdokumentu
      \lstinputlisting{listings/find-array-struct.js}
    \end{itemize}
  \end{frame}

  \begin{frame}
    \frametitle{Operacje na wynikach wyszukiwania}
    \begin{itemize}
      \item \texttt{db.users.find().sort(\{ klucz: kierunek \})} - gdzie kierunek to -1 lub 1
      \item \texttt{db.users.find().count()}
      \item \texttt{db.users.find().limit(5)}
      \begin{itemize}
        \item \texttt{db.users.findOne()} - tylko pierwszy wynik
        \item \texttt{db.users.find().limit(5).skip(10)}
      \end{itemize}
    \end{itemize}
  \end{frame}

  \begin{frame}
    \frametitle{Modyfikacja dokumentów}
    Dostępne operacje:
    \begin{itemize}
      \item \texttt{\$rename, \$set} - zmiana klucza/wartości
      \item \texttt{\$unset} - usunięcie pola
      \item \texttt{\$inc, \$mul}
      \item \texttt{\$min, \$max} - zmienia wartość tylko, jeśli poprzednia jest większa/mniejsza od podanej
      \item \texttt{\$push} - dodaje element do tablicy
      \item \texttt{\$pop} - usuwa pierwszy lub ostatni element tablicy
      \item \texttt{\$pull} - usuwa wszystkie elementy spełniające podany warunek
      \item Wiele innych: \url{http://docs.mongodb.org/manual/reference/operator/update/}
    \end{itemize}
  \end{frame}

  \begin{frame}
    \frametitle{Agregacja}
    MongoDB pozwala na wykonywanie operacji na danych za pomocą:
    \begin{itemize}
      \item Aggregation Framework
      \begin{itemize}
        \item Pozwala na wykonanie operacji określonych za pomocą predefiniowanych operatorów: \url{http://docs.mongodb.org/manual/reference/operator/aggregation/}
        \item Rozmiar danych wyjściowych ograniczony do 16MB
      \end{itemize}
      \item Map-reduce
      \begin{itemize}
        \item Operacje opisuje się w JavaScripcie - bardziej uniwersalne
        \item Brak ograniczenia na wielkość wyniku
        \item Mniej wydajne od Aggregation Framework
      \end{itemize}
      \item Predefiniowane operacje agregacji
      \begin{itemize}
        \item \texttt{count()}
        \item \texttt{distinct(POLE)}
        \item \texttt{group(...)}
        \item Szczegóły: \url{http://docs.mongodb.org/manual/core/single-purpose-aggregation/}
      \end{itemize}
    \end{itemize}
  \end{frame}

  \begin{frame}
    \frametitle{SQL vs MongoDB}
    \includegraphics[width=\textwidth,keepaspectratio]{img/mysql_vs_mongodb.jpg}
  \end{frame}

  \begin{frame}
    \frametitle{MongoDB w NodeJS}
    Używanie MongoDB w NodeJS nie różni się zbytnio od wykonywania operacji z konsoli:
    \lstinputlisting{listings/node.js}
    \url{http://mongodb.github.io/node-mongodb-native/}
  \end{frame}

  \begin{frame}
    \frametitle{Instalacja: Linux}
    Wystarczy zainstalować odpowiednią paczkę z repozytorium. Dla Ubuntu sprowadza się to do wykonania:

    \lstinputlisting{listings/install-linux.sh}
  \end{frame}

  \begin{frame}
    \frametitle{Instalacja: Windows}
	Tutaj jest to nieco bardziej skomplikowane, bo należy pobrać instalator dostępny pod adresem \url{https://www.mongodb.org/downloads}, uruchomić i przejść wszystkie kroki oraz w przypadku używania systemu Windows 7 lub Windows Server 2008 R2 należy również doinstalować poprawkę \url{http://support.microsoft.com/kb/2731284}
	
	Aby móc uruchomić bazę danych należy wcześniej utworzyć folder do przechowywania danych. Domyślną ścieżką jest \emph{C:$\backslash$data$\backslash$db} katalog możemy utworzyć z poziomu wiersza poleceń:
	
	 \lstinputlisting{listings/data.bat}
  \end{frame}

\end{document}
