\documentclass[10pt,a4paper]{article}

\usepackage{amsmath}
\usepackage{amsfonts}
\usepackage{amssymb}
\usepackage{graphicx}
\usepackage{listings}
\usepackage{color}

\definecolor{lightgray}{rgb}{.9,.9,.9}
\definecolor{darkgray}{rgb}{.4,.4,.4}
\definecolor{darkgreen}{rgb}{0,0.4,0}

\lstdefinelanguage{JavaScript}{
    keywords={typeof, new, true, false, catch, function, return, null, catch, switch, var, if, in, while, do, else, case, break},
    keywordstyle=\color{blue}\bfseries,
    ndkeywords={class, export, boolean, throw, implements, import, this},
    ndkeywordstyle=\color{darkgray}\bfseries,
    identifierstyle=\color{black},
    sensitive=false,
    comment=[l]{//},
    morecomment=[s]{/*}{*/},
    commentstyle=\color{darkgreen}\ttfamily,
    stringstyle=\color{red}\ttfamily,
    morestring=[b]',
    morestring=[b]"
}

\lstset{
    language=JavaScript,
    backgroundcolor=\color{lightgray},
    extendedchars=true,
    basicstyle=\footnotesize\ttfamily,
    showstringspaces=false,
    showspaces=false,
    tabsize=2,
    breaklines=true,
    showtabs=false,
    captionpos=b
}

\usepackage{polski}
\usepackage[polish]{babel}
\usepackage[utf8]{inputenc}

\author{Dawid Czech, Marcin Radomski}
\title{MongoDB}

\begin{document}
\maketitle
\newpage

\section{Wstęp}
\subsection{Model danych}
Dane w MongoDB są przechowywane jako zbiór \emph{par klucz-wartość}, które są pogrupowane w \emph{dokumenty}. Wiele dokumentów tworzy \emph{kolekcję}, a zbiór kolekcji to właściwa \emph{baza}. Warto zauważyć, że ten sposób przechowywania danych nie wymaga, żeby kolekcje przechowywały dane o jakiejś określonej strukturze - każdy z jego elementów może być zupełnie inny.

Takie rozwiązanie sprawia, że mapowanie obiektów w językach programowania na odpowiadające im dokumenty jest bardzo proste, zwłaszcza w przypadku używanego w tym tutorialu JavaScriptu.

\subsubsection{Zalety}
\begin{itemize}
\item Bazę tworzy się łatwo
\item Kolekcje nie narzucają określonej struktury danych w dokumencie
\end{itemize}

\subsubsection{Wady}
\begin{itemize}
\item Trzeba dobrze zastanowić się nad strukturą danych przechowywanych w bazie. Żle zaprojektowana baza będzie działać dużo wolniej niż relacyjna!
\end{itemize}

\subsubsection{Dokument}
Dane znajdujące się w dokumencie są zapisywane jako obiekty w formacie BSON (Binary JSON), który definiuje nie tylko typy prymitywne jak lliczba czy tekst, ale również tablice czy... obiekty:

\begin{lstlisting}
// users/ayende
{
   "type": "user",
   "name": "ayende"
}

// blogs/1
{
    "type": "blog",
    "users": ["users/ayende"],
    "name": "Ayende @ Rahien",
    "tags": [".NET", "Architecture", "Databases"]
}
// posts/1
{
    "blog": "blogs/1",
    "title": "RavenDB",
    "content": "... content ...",
    "categories": ["Raven", "NoSQL"]
    "tags" : ["RavenDB", "Announcements"],
    "comments": [
        { "content": "Great News" }
    ]
}
\end{lstlisting}

Dzięki zastosowaniu takiego rozwiązania dobrze zaprojektowana baza MongoDB pozwala na niezwykle szybki dostęp do danych. O ile trzymamy wszystkie dane, do których się odwołujemy w jednym miejscu, wystarczy odwołać się do jednego miejsca w bazie, zamiast odczytywać po kawałku z różnych miejsc, jak to często jest w bazach relacyjnych.

\section{Przykłady}
\subsection{Dobry}
\subsection{Zły}
\subsection{Jeszcze jeden?}

\end{document}
